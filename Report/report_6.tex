\documentclass[a4paper,12pt,titlepage,finall]{article}

\usepackage[T1,T2A]{fontenc}     % форматы шрифтов
\usepackage[russian]{babel}      % пакет русификации
\usepackage{tikz}                % для создания иллюстраций
\usepackage{pgfplots}            % для вывода графиков функций
\usepackage{geometry}		 % для настройки размера полей
\usepackage{indentfirst}         % для отступа в первом абзаце секции
\usepackage{multirow}
\usepackage{fontspec}
\usepackage{tikz}  
\usepackage{fancyvrb}
\usetikzlibrary{positioning,arrows}
% выбираем размер листа А4, все поля ставим по 3см
\geometry{a4paper,left=30mm,top=30mm,bottom=30mm,right=30mm}

\setmainfont{Times New Roman}
\setmonofont{Courier New}



\setcounter{secnumdepth}{0}      % отключаем нумерацию секций

\usepgfplotslibrary{fillbetween} % для изображения областей на графиках

\begin{document}
% Титульный лист
\begin{titlepage}
    \begin{center}
	{\small \sc Московский государственный университет \\имени М.~В.~Ломоносова\\
	Факультет вычислительной математики и кибернетики\\}
	\vfill
	{\Large \sc Отчет по заданию №6}\\
  ~\\
	{\large \bf <<Сборка многомодульных программ. \\
	Вычисление корней уравнений и определенных интегралов.>>}\\ 
	~\\
	{\large \bf Вариант 11 / 1 / 3}
    \end{center}
    \begin{flushright}
	\vfill {Выполнил:\\
	студент 103 группы\\
	Никифоров~Н.~И.\\
	~\\
	Преподаватель:\\
	Кузьменкова~Е.~А.}
    \end{flushright}
    \begin{center}
	\vfill
	{\small Москва\\2018}
    \end{center}
\end{titlepage}

% Автоматически генерируем оглавление на отдельной странице
\tableofcontents
\newpage

\section{Постановка задачи}

В данной работе необходимо было реализовать программу использующую вычеслительные методы для вычисления площади ограниченной фигуры тремя кривыми. Для вычисления корней использовался метод \"вилки\", а для вычисления интегралов метод Симсона. Программа реализовывалась на двух языках программирования C и Assembly. Причём решался сложный вариант задания, когда ассемблерный код генерировался отдельной программой, получающей заданные функции в файле вместе с отрезком, на котором нужно было посчитать искомую площадь.

\newpage

\section{Математическое обоснование}

Для тестирования программы был взят набор следующих функций(рис.~\ref{plot1}):
\begin{itemize}
  \item {$2-\tan{\frac{x}{4}}$}
  \item {$x$}
  \item {$\pi*0.2$}
\end{itemize}

Значения $\varepsilon_1$ и $\varepsilon_2$ вычислялись следующим образом: поскольку каждый корень вычислялся с погрешностью $\varepsilon_1$, при вычисление интеграла мы дополнительно ошибаемся максимум на $2*\varepsilon_1*max{f(x)}$. Следовательно наша итоговая погрешность должна быть меньше чем погрешность вычисления полощади, которая по условию должна быть равна 0.001. Следовательно $\varepsilon \geq 2*(\varepsilon_2 + 2*\varepsilon_1*max{f(x)})$. Подберём такие $\varepsilon_1$ и $\varepsilon_2$: так как $max{f(x)} \leq 10$ в условиях данной задачи, а значит можем взять $\varepsilon_1=0.00001$ и $\varepsilon_2=0.0001$. Тогда $0.001 \geq 2*(0.0001+2*0.00001*10) \Longrightarrow 0.001 \geq 0.0006$~\cite{math}. В итоге:
\begin{itemize}
  \item $\varepsilon_1=0.00001$
  \item $\varepsilon_2=0.0001$
\end{itemize}

\begin{figure}[h]
\centering
\begin{tikzpicture}
\begin{axis}[% grid=both,                % рисуем координатную сетку (если нужно)
             axis lines=middle,          % рисуем оси координат в привычном для математики месте
             restrict x to domain=0:4,   % задаем диапазон значений переменной x
             restrict y to domain=-1:6,  % задаем диапазон значений функции y(x)
             axis equal,                 % требуем соблюдения пропорций по осям x и y
             enlargelimits,              % разрешаем при необходимости увеличивать диапазоны переменных
             legend cell align=left,     % задаем выравнивание в рамке обозначений
             scale=2]                    % задаем масштаб 2:1

% первая функция
% параметр samples отвечает за качество прорисовки
\addplot[green,samples=256,thick] {2-tan(deg(x/4))};
% описание первой функции
\addlegendentry{$y=2-\tan{\frac{x}{4}}$}

% добавим немного пустого места между описанием первой и второй функций
\addlegendimage{empty legend}\addlegendentry{}

% вторая функция
% здесь необходимо дополнительно ограничить диапазон значений переменной x
\addplot[blue,samples=256,thick] {x};
\addlegendentry{$y=x$}

% дополнительное пустое место не требуется, так как формулы имеют небольшой размер по высоте

% третья функция
\addplot[red,samples=256,thick] {pi*0.2};
\addlegendentry{$y=\pi*0.2$}
\end{axis}
\end{tikzpicture}
\caption{Плоская фигура, ограниченная графиками заданных уравнений}
\label{plot1}
\end{figure}

\newpage

\section{Результаты экспериментов}

Для указанных в предыдущем разделе функций были найденны точки пересечения(таблица~\ref{table1}). Также была найдена площадь фигуры ограниченной данными кривыми. {$S=1.652$}

\begin{table}[h]
\centering
\begin{tabular}{|c|c|c|}
\hline
Кривые & $x$ & $y$ \\
\hline
1 и 2 &  1.5824 & 1.5828 \\
1 и 3 &  3.7634 & 0.6283 \\
2 и 3 &  0.6283 & 0.6283 \\
\hline
\end{tabular}
\caption{Координаты точек пересечения}
\label{table1}
\end{table}

Также результаты вычислений проиллюстрированы графически (рис.~\ref{plot2}).

\begin{figure}[h]
\centering
\begin{tikzpicture}
\begin{axis}[% grid=both,                % рисуем координатную сетку (если нужно)
             axis lines=middle,          % рисуем оси координат в привычном для математики месте
             restrict x to domain=0:4,  % задаем диапазон значений переменной x
             restrict y to domain=0:6,  % задаем диапазон значений функции y(x)
             axis equal,                 % требуем соблюдения пропорций по осям x и y
             enlargelimits,              % разрешаем при необходимости увеличивать диапазоны переменных
             legend cell align=left,     % задаем выравнивание в рамке обозначений
             scale=2,                    % задаем масштаб 2:1
             xticklabels={,,},           % убираем нумерацию с оси x
             yticklabels={,,}]           % убираем нумерацию с оси y



\addplot[green,samples=256,thick,name path=A] {2-tan(deg(x/4))};
% описание первой функции
\addlegendentry{$y=2-\tan{\frac{x}{4}}$}

% добавим немного пустого места между описанием первой и второй функций
\addlegendimage{empty legend}\addlegendentry{}

% вторая функция
% здесь необходимо дополнительно ограничить диапазон значений переменной x
\addplot[blue,samples=256,thick,name path=B] {x};
\addlegendentry{$y=x$}

% дополнительное пустое место не требуется, так как формулы имеют небольшой размер по высоте

% третья функция
\addplot[red,samples=256,thick,name path=C] {pi*0.2};
\addlegendentry{$y=\pi*0.2$}

% закрашиваем фигуру
\addplot[blue!20,samples=512] fill between[of=A and C,soft clip={domain=1.5824:3.7634}];
\addplot[blue!20,samples=512] fill between[of=B and C,soft clip={domain=0.6283:1.5825}];
\addlegendentry{$S=1.6516$}

% Поскольку автоматическое вычисление точек пересечения кривых в TiKZ реализовать сложно,
% будем явно задавать координаты.
\addplot[dashed] coordinates {(0.6283, 0.6283) (0.6283, 0)};
\addplot[color=black] coordinates {(0.6283, 0)} node [label={-90:{\small 0.6283}}]{};

\addplot[dashed] coordinates {(1.5824, 1.5824) (1.5824, 0)};
\addplot[color=black] coordinates {(1.5824, 0)} node [label={-90:{\small 1.5824}}]{};

\addplot[dashed] coordinates {(3.7634, 0.6283) (3.7634, 0)};
\addplot[color=black] coordinates {(3.7634, 0)} node [label={-90:{\small 3.7634}}]{};

\end{axis}
\end{tikzpicture}
\caption{Плоская фигура, ограниченная графиками заданных уравнений}
\label{plot2}
\end{figure}

\newpage

\section{Структура программы и спецификация функций}

Программа для генерации ассемблерного кода состоит из трёх модулей.
\begin{enumerate}
  \item Первый главный модуль этой программы состоит из функций:
  \begin{itemize}
    \item {\bf \ttfamily void bind\_fun\_data(function *f1, int *offest)} функция пишет в ассемблерный файл в секцию rodata константы, которые использует функция.
    \item {\bf \ttfamily void bind\_section\_data(function *f1, function *f2, \newline function *f3, double a, double b)} оболочка для предыдущей функции, которая использует её для записи в файл констант всех функций, так же она пишет в секцию rodata границы отрезка и константу e.
    \item {\bf \ttfamily void bind\_function(function *fun, int counter)} функция пишет в ассемблерный файл функцию, поданную ей в качестве аргумента ввиде масива структур.
  \end{itemize}

  \item Второй модуль используется для парсинга текстового файла с исходными функциями. В нём представленны следующие функции:
  \begin{itemize}
    \item {\bf \ttfamily int parse\_sym(char now)} функция возращает тип поданного на фход символа.
    \item {\bf \ttfamily void read\_fun(FILE *file, mem *fun)} функция разбирает строчку из файла интерпретируя её как обратную польскую запись и записывает её в массив логических элементов.
    \item {\bf \ttfamily void read\_file(FILE *file, mem *f1, mem *f2, mem *f3, \newline double *a, double *b)} функция является оболочкой для предыдущей функции и читает из файла три функии.
    \item {\bf \ttfamily void print\_fun(mem *fun)} функция печатает массив логических элементов функции.
  \end{itemize}
  
  \item Третий модуль используется для записи базовых функций сопроцессора в ассемблерный файл. В нём содержится класс функций {\bf \ttfamily void bind\_*()} где * -- название функции сопроцессора, соответственно функция пишет соответствующую функцию сопроцессора в ассемблерный файл.

\end{enumerate}

Главная программа состоит из двух модулей, на C и Assembly.
\begin{enumerate}
  \item Первый модуль отвечает за подсчёт площади ограниченной тремя кривыми и содержит следующие функции:
  \begin{itemize}
    \item {\bf \ttfamily int process\_flags(char *flg)} функция занимающаяся парсингом флагов поданных программе на вход.
    \item {\bf \ttfamily double ras\_fun(doube fun1(double), double fun2(double),\newline double now)} функция считает разность значений функций поданных на вход.
    \item {\bf \ttfamily double mod(double a)} функция считающая модуль вещественного числа.
    \item {\bf \ttfamily double root(double fun1(double), double fun2(double),\newline double a, double b, double eps)} функция для поиска корней уравнения {\bf \ttfamily fun1 = fun2} на отрезке [a, b] c точностью eps c помощью метода вилки
    \item {\bf \ttfamily integral(double fun(double), double a, double b, double eps)} функция считающая интеграл от поданной на вход функции на заданном отрезке [a, b] с точностью eps.
    \item {\bf \ttfamily double f1f2(double x)} функция считающая разность функций f1 и f2 в точке x.
    \item {\bf \ttfamily double f1f3(double x)} функция считающая разность функций f1 и f3 в точке x.
    \item {\bf \ttfamily double f2f3(double x)} функция считающая разность функций f2 и f3 в точке x.
    \item {\bf \ttfamily double zero\_fun()} функция тождественно равная нулю.
  \end{itemize}
  
  \item Второй модуль автоматически генерируется первой программой на си. Он реализует данные функции на ассемблере, соответственно:
  \begin{itemize}
    \item {\bf \ttfamily double f1(double x)} вычисляет f1 в точке x.
    \item {\bf \ttfamily double f2(double x)} вычисляет f2 в точке x.
    \item {\bf \ttfamily double f3(double x)} вычисляет f3 в точке x.
    \item {\bf \ttfamily double get\_a()} возвращает левую границу отрезка.
    \item {\bf \ttfamily double get\_b()} возвращает правую границу отрезка.
  \end{itemize}
\end{enumerate}

\begin{tikzpicture}[thick,
                    node distance=4cm,
                    text height=1.5ex]

  \tikzstyle{format} = [thick, minimum size=1cm, draw=black!50!black!50]
  \node[format] (main) {main.c};
  \node[format,right of=main] (funct) {functions.asm};

  \path[->] (main) edge (funct);
\end{tikzpicture}


\begin{tikzpicture}[thick,
                    node distance=4cm,
                    text height=1.5ex]

  \tikzstyle{format} = [thick, minimum size=1cm, draw=black!50!black!50]
  \node[format] (genasm) {gen\_asm.c};
  \node[format,right of=genasm] (bindcom) {bind\_commands.c};
  \node[format,left of=genasm] (parsefun) {parse\_fun.c};

  \path[->] (genasm) edge (bindcom);
  \path[->] (genasm) edge (parsefun);
\end{tikzpicture}

\newpage

\section{Сборка программы (Make-файл)}

В данном разделе необходимо описать зависимости между модулями программы
и привести текст Make-файла. Зависимости проще всего описать диаграммой.

\begin{Verbatim}
CXX=gcc
CXXFLAGS=-std=c99 -O2
ADDITIONALFLG=-c

all: main

main: asm functions.o main.c
	${CXX} ${CXXFLAGS} functions.o  main.c -o main -m32 -lm

asm: gen_asm ${SPEC_FILE}
	./gen_asm ${SPEC_FILE} > functions.asm
	nasm -f elf32 functions.asm -o functions.o -D UNIX

gen_asm: gen_asm.c parse_fun.o bind_commands.o
	${CXX} ${CXXFLAGS} gen_asm.c parse_fun.o bind_commands.o -o gen_asm

parse_fun.o: parse_fun.c parse_fun.h
	${CXX} ${CXXFLAGS} ${ADDITIONALFLG} parse_fun.c 

bind_commands.o: bind_commands.c bind_commands.h
	${CXX} ${CXXFLAGS} ${ADDITIONALFLG} bind_commands.c

clean: 
	rm functions.asm main gen_asm *.o
\end{Verbatim}


\begin{tikzpicture}[thick,
                    node distance=4cm,
                    text height=1.5ex]

  \tikzstyle{format} = [thick, minimum size=1cm, draw=black!50!black!50]
  \node[format] (main) {main};
  \node[format,right of=main] (asm) {asm};
  \node[format,right of=asm] (genasm) {gen\_asm};
  \node[format,right of=genasm] (parsefun) {parse\_fun.o};
  \node[format,below of=parsefun] (bindcom) {bind\_commands.o};
  \node[format,below of=main] (clean) {clean};
  \node[format,below of=asm] (specfile) {SPEC\_FILE};

  \path[->] (genasm) edge (bindcom);
  \path[->] (genasm) edge (parsefun);
  \path[->] (main) edge (asm);
  \path[->] (asm) edge (genasm);
  \path[->] (asm) edge (specfile);
\end{tikzpicture}

\newpage

\section{Отладка программы, тестирование функций}
Тестирование программы производилось с помощью специально реализованных флагов {\ttfamily -testRoot -testInteg} для тестирования функции вычисления корней и интегралов соответственно. Для каждого флага программа компилировалась со специальным значением {\ttfamily SPEC\_FILE}, в котором для тестирования каждой из функций были свои функции. Проверка правильности вычисления корней и интегралов проводилась при помощи аналитического вычисления и дополнительной проверки с помолью Wolfram alpha.

\noindent Для тестировария функции корней использовались следующие функции:
\begin{enumerate}
  \item {$x+2=0$} результаты тестирования для данной функции на отрезке [0, 4]:
  \begin{itemize}
    \item Результат программы: 2.00000
    \item Результат аналитических вычислений: 2
    \item Результат Wolfram alpha: 2.0000
  \end{itemize}
  \item {$(x-3)^2=0$} результаты тестирования для данной функции на отрезке [0, 4]:
  \begin{itemize}
    \item Результат программы: 1.732100
    \item Результат аналитических вычислений: {$\sqrt{3}$}
    \item Результат Wolfram alpha: 1.7321
  \end{itemize}
\item {$x^4+x^2-4$=0} результаты тестирования для данной функции на отрезке [0, 4]:
  \begin{itemize}
    \item Результат программы: 1.249600
    \item Результат аналитических вычислений: {$\sqrt{\frac{\sqrt{17}}{2}-\frac{1}{2}}$}
    \item Результат Wolfram alpha: 1.2496
  \end{itemize}
\end{enumerate}
Таким образом результаты тестирования функции исчисления корней совпадают с аналитичискими вычислениями.
\newpage
\noindent Для тестирования функции интегралов использовались следующие функции:
\begin{enumerate}
  \item {$x^2+2$} результаты тестирования для данной функции на отрезке [0, 4]:
  \begin{itemize}
    \item Результат программы: 29.333300
    \item Результат аналитических вычислений: {$\int_0^4 \!(x^2+2)\, \mathrm{d}x=\frac{4^3}{3}+2*4=29\frac{1}{3}$}
    \item Результат Wolfram alpha: 29.3333
  \end{itemize}
  \item {$x^3$} результаты тестирования для данной функции на отрезке [0, 4]:
  \begin{itemize}
    \item Результат программы: 64.00000
    \item Результат аналитических вычислений: {$\int_0^4\!x^3\,\mathrm{d}x=64$}
    \item Результат Wolfram alpha: 64
  \end{itemize}
\item {$\sin(x)+1$} результаты тестирования для данной функции на отрезке [0, 4]:
  \begin{itemize}
    \item Результат программы: 5.653700
    \item Результат аналитических вычислений: {$\int_0^4\!\sin(x)+1\,\mathrm{d}x=4-\cos(4)+\cos(0)=5.6536$}
    \item Результат Wolfram alpha: 5.6536
  \end{itemize}
\end{enumerate}

Таким образом результаты тестирования функции исчисления интегралов совпадают с аналитическими вычислениями.

\newpage

\section{Программа на C и Assembly}

Программа реализовывалась на двух языках C и Assembly, причём ассемблерная часть генерировалась отдельной программой также написанной на C. Все тексты программ приложены к отчёту в архиве. 
\newpage

\section{Анализ допущенных ошибок}

\newpage
\begin{raggedright}
\addcontentsline{toc}{section}{Список цитируемой литературы}
\begin{thebibliography}{99}
\bibitem{math} Ильин~В.\,А., Садовничий~В.\,А., Сендов~Бл.\,Х. Математический анализ. Т.\,1~---
    Москва: Наука, 1985.
\end{thebibliography}
\end{raggedright}

\end{document}
